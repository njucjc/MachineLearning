\documentclass[a4paper,UTF8]{article}
\usepackage{ctex}
\usepackage[margin=1.25in]{geometry}
\usepackage{color}
\usepackage{graphicx}
\usepackage{amssymb}
\usepackage{amsmath}
\usepackage{amsthm}
\usepackage{enumerate}
\usepackage{bm}
\usepackage{hyperref}
\usepackage{epsfig}
\usepackage{color}
\usepackage{mdframed}
\usepackage{lipsum}
\usepackage{mathtools}
\usepackage{hyperref}
\newmdtheoremenv{thm-box}{myThm}
\newmdtheoremenv{prop-box}{Proposition}
\newmdtheoremenv{def-box}{定义}

\usepackage{algorithm}  
\usepackage{algorithmicx}  
\usepackage{algpseudocode} 
\floatname{algorithm}{算法}  
\renewcommand{\algorithmicrequire}{\textbf{输入:}}  
\renewcommand{\algorithmicensure}{\textbf{输出:}}  

\usepackage{listings}
\usepackage{xcolor}
\lstset{
	numbers=left, 
	numberstyle= \tiny, 
	keywordstyle= \color{ blue!70},
	commentstyle= \color{red!50!green!50!blue!50}, 
	frame=shadowbox, % 阴影效果
	rulesepcolor= \color{ red!20!green!20!blue!20} ,
	escapeinside=``, % 英文分号中可写入中文
	xleftmargin=2em,xrightmargin=2em, aboveskip=1em,
	framexleftmargin=2em
} 

\usepackage{booktabs}

\setlength{\evensidemargin}{.25in}
\setlength{\textwidth}{6in}
\setlength{\topmargin}{-0.5in}
\setlength{\topmargin}{-0.5in}
% \setlength{\textheight}{9.5in}
%%%%%%%%%%%%%%%%%%此处用于设置页眉页脚%%%%%%%%%%%%%%%%%%
\usepackage{fancyhdr}                                
\usepackage{lastpage}                                           
\usepackage{layout}                                             
\footskip = 10pt 
\pagestyle{fancy}                    % 设置页眉                 
\lhead{2018年秋季}                    
\chead{高级机器学习}                                                
% \rhead{第\thepage/\pageref{LastPage}页} 
\rhead{作业一}                                                                                               
\cfoot{\thepage}                                                
\renewcommand{\headrulewidth}{1pt}  			%页眉线宽,设为0可以去页眉线
\setlength{\skip\footins}{0.5cm}    			%脚注与正文的距离           
\renewcommand{\footrulewidth}{0pt}  			%页脚线宽,设为0可以去页脚线

\makeatletter 									%设置双线页眉                                        
\def\headrule{{\if@fancyplain\let\headrulewidth\plainheadrulewidth\fi%
\hrule\@height 1.0pt \@width\headwidth\vskip1pt	%上面线为1pt粗  
\hrule\@height 0.5pt\@width\headwidth  			%下面0.5pt粗            
\vskip-2\headrulewidth\vskip-1pt}      			%两条线的距离1pt        
 \vspace{6mm}}     								%双线与下面正文之间的垂直间距              
\makeatother  

%%%%%%%%%%%%%%%%%%%%%%%%%%%%%%%%%%%%%%%%%%%%%%
\numberwithin{equation}{section}
%\usepackage[thmmarks, amsmath, thref]{ntheorem}
\newtheorem{theorem}{Theorem}
\newtheorem*{definition}{Definition}
\newtheorem*{solution}{Solution}
\newtheorem*{prove}{Proof}
\newcommand{\indep}{\rotatebox[origin=c]{90}{$\models$}}

\usepackage{multirow}

%--

%--
\begin{document}
\title{高级机器学习\\
作业一}
\author{学号:MF1033004, 作者姓名:陈锦赐, 邮箱:jimchen@smail.nju.edu.cn}
\maketitle

\section{[25pts] Multi-Class Logistic Regression}
教材的章节3.3介绍了对数几率回归解决二分类问题的具体做法。假定现在的任务不再是二分类问题,而是多分类问题,其中$y\in\{1,2\dots,K\}$。请将对数几率回归算法拓展到该多分类问题。

\begin{enumerate}
	\item \textbf{[15pts]} 给出该对率回归模型的“对数似然”(log-likelihood);
	\item \textbf{[5pts]} 计算出该“对数似然”的梯度。
\end{enumerate}

提示1:假设该多分类问题满足如下$K-1$个对数几率,
\begin{eqnarray*}
	\ln\frac{p(y=1|\mathbf{x})}{p(y=K|\mathbf{x})}&=&\mathbf{w}_1^\mathrm{T}\mathbf{x}+b_1\\
	\ln\frac{p(y=2|\mathbf{x})}{p(y=K|\mathbf{x})}&=&\mathbf{w}_2^\mathrm{T}\mathbf{x}+b_2\\
	&\dots&\\
	\ln\frac{p(y={K-1}|\mathbf{x})}{p(y=K|\mathbf{x})}&=&\mathbf{w}_{K-1}^\mathrm{T}\mathbf{x}+b_{K-1}
\end{eqnarray*}

提示2:定义指示函数$\mathbb{I}(\cdot)$,
$$\mathbb{I}(y=j)=
\begin{cases}
1& \text{若$y$等于$j$}\\
0& \text{若$y$不等于$j$}
\end{cases}$$

\begin{solution}
此处用于写解答(中英文均可)
\begin{enumerate}
	\item 对于所有的K个可能的分类结果,运用$K-1$个独立的二元对数几率回归模型建模,即把其中一个类别看作主类别,然后将其他$K-1$个类别和选取的主类别分别进行对数几率回归。据此,若将类别$K$作为主类别,可以得到如下$K-1$个式子:
		\begin{eqnarray}
			\ln\frac{p(y=1|\mathbf{x})}{p(y=K|\mathbf{x})}&=&\mathbf{w}_1^\mathrm{T}\mathbf{x}+b_1\\
			\ln\frac{p(y=2|\mathbf{x})}{p(y=K|\mathbf{x})}&=&\mathbf{w}_2^\mathrm{T}\mathbf{x}+b_2 \nonumber\\
			&\dots& \nonumber\\
			\ln\frac{p(y={K-1}|\mathbf{x})}{p(y=K|\mathbf{x})}&=&\mathbf{w}_{K-1}^\mathrm{T}\mathbf{x}+b_{K-1} \nonumber
		\end{eqnarray}
		对式子$(1.1)$的左右两边进行指数化,可以得到如下$K-1$个式子:
		\begin{eqnarray}
			p(y=1|\mathbf{x})&=&p(y=K|\mathbf{x})e^{\mathbf{w}_1^\mathrm{T}\mathbf{x}+b_1}\\
			p(y=2|\mathbf{x})&=&p(y=K|\mathbf{x})e^{\mathbf{w}_2^\mathrm{T}\mathbf{x}+b_2} \nonumber\\
			&\dots& \nonumber\\
			p(y={K-1}|\mathbf{x})&=&p(y=K|\mathbf{x})e^{\mathbf{w}_{K-1}^\mathrm{T}\mathbf{x}+b_{K-1}} \nonumber
		\end{eqnarray}
		注意到最终所有类别的概率加起来为1,故将式子$(1.2)$左右相加,我们可以得到如下式子:
		\begin{eqnarray}
			p(y=K|\mathbf{x})&=&1-\sum_{k=1}^{K-1}p(y=K|\mathbf{x})e^{\mathbf{w}_k^\mathrm{T}\mathbf{x}+b_k}\\
			&=&\frac{1}{1+\sum_{k=1}^{K-1} e^{\mathbf{w}_k^\mathrm{T}\mathbf{x}+b_k}}\nonumber
		\end{eqnarray}
		将式子$(1.3)$代入式子$(1.2)$得到如下$K-1$个式子:
		\begin{eqnarray}
			p(y=1|\mathbf{x})&=&\frac{e^{\mathbf{w}_1^\mathrm{T}\mathbf{x}+b_1}}{1+\sum_{k=1}^{K-1} e^{\mathbf{w}_k^\mathrm{T}\mathbf{x}+b_k}}\\
			p(y=2|\mathbf{x})&=&\frac{e^{\mathbf{w}_2^\mathrm{T}\mathbf{x}+b_2}}{1+\sum_{k=1}^{K-1} e^{\mathbf{w}_k^\mathrm{T}\mathbf{x}+b_k}} \nonumber\\
			&\dots& \nonumber\\
			p(y={K-1}|\mathbf{x})&=&\frac{e^{\mathbf{w}_{K-1}^\mathrm{T}\mathbf{x}+b_{K-1}}}{1+\sum_{k=1}^{K-1} e^{\mathbf{w}_k^\mathrm{T}\mathbf{x}+b_k}} \nonumber
		\end{eqnarray}
		%为了方便讨论,令$\mathbf{\beta}_i=(\mathbf{w}_i;b_i)$,$\mathbf{\widehat{x}}=(\mathbf{x};1)$
		定义指示函数:
		$$\mathbb{I}(y=j)=
		\begin{cases}
		1& \text{若$y$等于$j$}\\
		0& \text{若$y$不等于$j$}
		\end{cases}$$
		给定数据集为$\{(\mathbf{x_i}, y_i)\}_{i=1}^m$,则其对数似然为:
		\begin{eqnarray}
			l(\mathbf{w}, b) &=& \sum_{i=1}^m\ln p(y_i|\mathbf{x_i})
		\end{eqnarray}
		其中:
		\begin{eqnarray}
			p(y_i|\mathbf{x_i})&=&\sum_{j=1}^K\mathbb{I}(y_i=j)p(y_i=j|\mathbf{x_i})\\
			&=&\sum_{j=1}^{K-1}\mathbb{I}(y_i = j)\frac{e^{\mathbf{w}_j^\mathrm{T}\mathbf{x}_i+b_j}}{1+\sum_{j=1}^{K-1}e^{\mathbf{w}_j^\mathrm{T}\mathbf{x}_i+b_j}}+\mathbb{I}(y_i = j)\frac{1}{1+\sum_{j=1}^{K-1}e^{\mathbf{w}_j^\mathrm{T}\mathbf{x}_i+b_j}}\nonumber
		\end{eqnarray}
		将$(1.6)$代回$(1.5)$得:
		\begin{eqnarray}
			l(\mathbf{w}, b) &=& \sum_{i=1}^m\ln p(y_i|\mathbf{x_i})\\
			&=&\sum_{i=1}^m\ln[\sum_{j=1}^{K-1}\mathbb{I}(y_i=j)\frac{e^{\mathbf{w}_j^\mathrm{T}\mathbf{x}_i+b_j}}{1+\sum_{j=1}^{K-1}e^{\mathbf{w}_j^\mathrm{T}\mathbf{x}_i+b_j}}+\mathbb{I}(y_i=j)\frac{1}{1+\sum_{j=1}^{K-1}e^{\mathbf{w}_j^\mathrm{T}\mathbf{x}_i+b_j}}]\nonumber\\
			&=&\sum_{i=1}^m\ln[\frac{\sum_{j=1}^{K-1}\mathbb{I}(y_i=j)e^{\mathbf{w}_j^\mathrm{T}\mathbf{x}_i+b_j}+\mathbb{I}(y_i=j)}{1+\sum_{j=1}^{K-1}e^{\mathbf{w}_j^\mathrm{T}\mathbf{x}_i+b_j}}]\nonumber\\
			&=&\sum_{i=1}^m(\ln[\sum_{j=1}^{K-1}\mathbb{I}(y_i=j)e^{\mathbf{w}_j^\mathrm{T}\mathbf{x}_i+b_j}+\mathbb{I}(y_i=j)]-\ln[1+\sum_{j=1}^{K-1}e^{\mathbf{w}_j^\mathrm{T}\mathbf{x}_i+b_j}])\nonumber
		\end{eqnarray}
		又因为$\sum_{j=1}^K\mathbb{I}(y_i=j)=1$,所以:
		\begin{eqnarray}
			\ln[\sum_{j=1}^{K-1}\mathbb{I}(y_i=j)e^{\mathbf{w}_j^\mathrm{T}\mathbf{x}_i+b_j}+\mathbb{I}(y_i=j)*1]&=&\sum_{j=1}^{K-1}\mathbb{I}(y_i=j)\ln[e^{\mathbf{w}_j^\mathrm{T}\mathbf{x}_i+b_j}]\\
			&=&\sum_{j=1}^{K-1}\mathbb{I}(y_i=j)(\mathbf{w}_j^\mathrm{T}\mathbf{x}_i+b_j)\nonumber
		\end{eqnarray}
		将$(1.8)$代回$(1.7)$得
		\begin{eqnarray}
			l(\mathbf{w}, b) &=& \sum_{i=1}^m(\sum_{j=1}^{K-1}\mathbb{I}(y_i=j)(\mathbf{w}_j^\mathrm{T}\mathbf{x}_i+b_j)-\ln[1+\sum_{j=1}^{K-1}e^{\mathbf{w}_j^\mathrm{T}\mathbf{x}_i+b_j}])
		\end{eqnarray}
		式子$(1.9)$即为所得。
	\item 为了方便描述,我们令$\beta_j=(\mathbf{w}_j;b_j),\widetilde{\mathbf{x}}=(\mathbf{x};1)$,则式$(1.9)$可改写为:
	\begin{eqnarray}
		l(\beta) &=& \sum_{i=1}^m(\sum_{j=1}^{K-1}\mathbb{I}(y_i=j)\beta_j^\mathrm{T}\widetilde{\mathbf{x}}_i-\ln[1+\sum_{j=1}^{K-1}e^{\beta_j^\mathrm{T}\widetilde{\mathbf{x}}_i}])
	\end{eqnarray}
	所以,求其梯度为:
	\begin{eqnarray}
		\bigtriangledown l(\beta) &=& \sum_{i=1}^m(\sum_{j=1}^{K-1}\mathbb{I}(y_i=j)\widetilde{\mathbf{x}}_i-\frac{\sum_{j=1}^{K-1}\widetilde{\mathbf{x}}_i e^{\beta_j^\mathrm{T}\widetilde{\mathbf{x}}_i}}{1+\sum_{j=1}^{K-1}e^{\beta_j^\mathrm{T}\widetilde{\mathbf{x}}_i}})
	\end{eqnarray}

\end{enumerate}
\end{solution}
\newpage

\section{[15pts] Semi-Supervised Learning}
我们希望使用半监督学习的方法来对文本文档进行分类。假设我们使用二进制指示符的词袋模型描述各个文档,在这里,我们的词库有$10000$个单词,因此每个文档由长度为$10000$的二进制向量表示。

对于以下提出的分类器,说明其是否可以用于改进学习性能并提供简要说明。
\begin{enumerate}
	\item \textbf{[5pts]} 使用EM的朴素贝叶斯;
	\item \textbf{[5pts]} 使用协同训练的朴素贝叶斯;
	\item \textbf{[5pts]} 使用基于特征选择的朴素贝叶斯;
\end{enumerate}

\begin{solution}
此处用于写解答(中英文均可)
	\begin{enumerate}
		\item 使用EM的朴素贝叶斯能够提升学习性能,说明如下:

		基于EM的朴素贝叶斯首先仅利用带分类标签的文档训练出一个朴素贝叶斯分类器,然后利用EM算法迭代地使用全部文档训练优化参数并得到一个新的分类器。EM算法根据初始模型的参数空间,迭代地执行:
		\begin{enumerate}[(1)]
			\item E(期望)步:利用当前的参数空间对未标记的文档进行分离,获得未标记文档的标签,这样能够求出当前参数条件下所有数据的联合分布期望。
			\item M(最大化期望)步:利用带分类标签的文档和刚刚标记好类别的文档作为完全数据求解先验概率分布和条件概率分布作为新的参数空间,再执行E步,直至收敛。
		\end{enumerate}

		EM算法是爬山算法,能够保证每一步迭代使结果得到改善,最终得到一个局部最优解。

		\item 使用协同训练的朴素贝叶斯能够提升学习性能,说明如下:

		协同训练假设练数据集存在两个视图,首先使用各个视图的有标记数据分别训练出一个分类器,然后让各个分类器在各自的视图的未标记数据中选择置信度最高的进行标注,并添加到对方的有标记样本集中进行训练,这个过程不断迭代,直到满足某个停止条件[3]。

		可以看出协同训练要求数据集能够找到两个充分冗余的视。但是这个条件在真实问题中很难满足,但根据K.Nigam等在文献[4]中的研究表明,在属性集充分大时,可以随机把属性集划分成两个视图,在此基础上进行协同训练也能够取得不错的结果。而在二进制指示符的词袋模型中,每个文档都由$1000$维的向量表示,基本满足属性集足够大的要求。
		\item 使用基于特征选择的朴素贝叶斯能够提升学习性能,说明如下:

		特征选择是指从高维文本(在本例中是$1000$维)的特征集合中挑选出一部分特征组成一个低维的向量空间的过程。特征选择基于一个事实:“并非数据维度越高分类的效果就越好”。高维的特征不仅增加了机器学习的负担,还提高了分类的计算复杂度,并且文献[5]指出,过高的特征维数反而有可能降低分类的准确性,形成“维度灾难”。

		在整个特征集合中,有很多词在各个类别的文档中出现的频率很相似,对类别区分的帮助不大。还有一些低频词,无法作为区分类别的参考。文本特征选择目标就是去除这些对区分类别没有作用的特征项。对文本进行降维处理,能够提升分类精度,而且可以减小计算复杂度[6]。


	\end{enumerate}
\end{solution}
\newpage

\section{[60pts] Dimensionality Reduction}
请实现三种降维方法:PCA,SVD和ISOMAP,并在降维后的空间上用$1$-NN方法分类。
\begin{enumerate}
	\item 数据:我们给出了两个数据集,都是二分类的数据。可以从\url{https://www.csie.ntu.edu.tw/~cjlin/libsvmtools/datasets/binary.html}找到,同时也可以在提交作业的目录文件夹中找名为“two datasets”的压缩文件下载使用。每个数据集都由训练集和测试集组成。
	\item 格式:再每个数据集中,每一行表示一个带标记的样例,即每行最后一列表示对应样例的标记,其余列表示对应样例的特征。
\end{enumerate}

具体任务描述如下:
\begin{enumerate}
	\item \textbf{[20pts]} 请实现PCA完成降维(方法可在参考书\url{http://www.charuaggarwal.net/Data-Mining.htm} 中 Section 2.4.3.1 中找到)
	\subitem 首先,仅使用训练数据学习投影矩阵;
	\subitem 其次,用学得投影矩阵将训练数据与测试数据投影到 $k$-维空间 ($k=10,20,30$);
	\subitem 最后,在降维后空间上用$1$-NN预测降维后$k$维数据对应的标记 ($k=10,20,30$),并汇报准确率。注意,测试数据集中的真实标记仅用来计算准确率。
	\item \textbf{[20pts]} 请实现SVD完成降维(方法在上述参考书 Section 2.4.3.2 中找到)
	\subitem 首先,仅使用训练数据学习投影矩阵;
	\subitem 其次,用学得投影矩阵将训练数据与测试数据投影到 $k$-维空间 ($k=10,20,30$);
	\subitem 最后,在降维后空间上用$1$-NN预测降维后$k$维数据对应的标记 ($k=10,20,30$),并汇报准确率。注意,测试数据集中的真实标记仅用来计算准确率。
	\item \textbf{[20pts]} 请实现ISOMAP完成降维(方法在参考书 Section 3.2.1.7 中找到)
	\subitem 首先,使用训练数据与测试数据学习投影矩阵。在这一步中,请用$4$-NN来构建权重图。(请注意此处$4$仅仅是用来举例的,可以使用其他 $k$-NN, $k\geq 4$并给出你选择的k。如果发现构建的权重图不连通,请查找可以解决该问题的方法并汇报你使用的方法)
	\subitem 其次,用学得投影矩阵将训练数据与测试数据投影到 $k$-维空间 ($k=10,20,30$)。
	\subitem 最后,在降维后空间上用$1$-NN预测降维后$k$维数据对应的标记 ($k=10,20,30$),并汇报准确率。注意,测试数据集中的真实标记仅用来计算准确率。
\end{enumerate}

可以使用已有的工具、库、函数等直接计算特征值和特征向量,执行矩阵的SVD分解,计算graph上两个节点之间的最短路。PCA/SVD/ISOMAP 和 $1$-NN 中的其他步骤必须由自己实现。

报告中需要包含三个方法的伪代码和最终的结果。最终结果请以表格形式呈现,表中包含三种方法在两个数据集中,不同 $k=10,20,30$ 下的准确率。
\newpage

\begin{solution}
	此处用于写解答(中英文均可)
	\begin{enumerate}
	\item PCA降维的结果如表$1$所示:
	\begin{table}[htbp]
  		\centering
  		\begin{tabular}{cccp{38mm}}
	    \toprule
	    \textbf{数据集} & \textbf{k=10} & \textbf{k=20} & \textbf{k=30} \\
	    \midrule
	    sonar  & 0.582524 & 0.563107 & 0.563107\\
	    splice & 0.758161 & 0.762759 & 0.735632\\
	    \bottomrule
  		\end{tabular}
  		\caption{PCA降维结果}\label{table:1}
	\end{table}

	其中PCA的伪代码为如算法$1$所示:
	   	\begin{algorithm}[htbp]  
	        \caption{PCA算法}  
	        \begin{algorithmic}[1] %每行显示行号  
	            \Require 训练集矩阵$\mathbf{X}=\{\mathbf{x_1}, \mathbf{x_2},\dots,\mathbf{x_m}\}$,低维空间维数$d$ 
	            \Ensure 投影矩阵$\mathbf{W}$及均值向量$\bar{\mathbf{x}}$  
	            \State 求均值向量:$\bar{\mathbf{x}} \gets \frac{1}{m}\sum_{i=1}^{m}\mathbf{x_i}$;
	            \State 对所有样本进行中心化:$\mathbf{x_i} \gets \mathbf{x_i} - \bar{\mathbf{x}}$;
	            \State 计算协方差矩阵$\mathbf{B}\mathbf{B}^T$;
	            \State 对协方差矩阵$\mathbf{B}\mathbf{B}^T$做特征值分解;
	            \State 取最大的d个特征值对应的特征向量组成投影矩阵$\mathbf{W}=\{\mathbf{w_1}, \mathbf{w_2},\dots,\mathbf{w_d}\}$;
	            \State \Return{$\mathbf{W},\bar{\mathbf{x}}$}
	        \end{algorithmic}  
    	\end{algorithm} 

    对新数据降维时,只需用均值向量进行中心化后乘以投影矩阵即可。
	\item SVD降维的结果如表$2$所示:
	\begin{table}[htbp]
  		\centering
  		\begin{tabular}{cccp{38mm}}
	    \toprule
	    \textbf{数据集} & \textbf{k=10} & \textbf{k=20} & \textbf{k=30} \\
	    \midrule
	    sonar  & 0.582524 & 0.563107 & 0.563107\\
	    splice & 0.758161 & 0.762759 & 0.735632\\
	    \bottomrule
  		\end{tabular}
  		\caption{SVD降维结果}\label{table:1}
	\end{table}

	其中SVD降维的伪代码如算法$2$所示:
	\begin{algorithm}[htbp]  
        \caption{SVD算法}  
        \begin{algorithmic}[1] %每行显示行号  
 			\Require 训练集矩阵$\mathbf{X}=\{\mathbf{x_1}, \mathbf{x_2},\dots,\mathbf{x_m}\}$,低维空间维数$d$ 
	        \Ensure 投影矩阵$\mathbf{W}$及均值向量$\bar{\mathbf{x}}$  
	        \State 求均值向量:$\bar{\mathbf{x}} \gets \frac{1}{m}\sum_{i=1}^{m}\mathbf{x_i}$;
	        \State 对所有样本进行中心化得到矩阵$\mathbf{C}$:$\mathbf{c_i} \gets \mathbf{x_i} - \bar{\mathbf{x}}$;
            \State 对矩阵$\mathbf{C}$进行奇异值分解:$\mathbf{C}=\mathbf{U}\mathbf{\sum}\mathbf{V^T}$;
            \State 令$\mathbf{W}$为$\mathbf{V^T}$的前$d$行组成的矩阵的转置;
            \State \Return{$\mathbf{W},\bar{\mathbf{x}}$}
        \end{algorithmic}  
    \end{algorithm} 

    对新数据降维时,只需用均值向量进行中心化后乘以投影矩阵即可。
\newpage
	\item ISOMAP降维的结果如表$3$所示(取$k=10$即$10$近邻,图不连通的问题采用利用一个较大的值代替进行无穷大近似计算):
	\begin{table}[htbp]
  		\centering
  		\begin{tabular}{cccp{38mm}}
	    \toprule
	    \textbf{数据集} & \textbf{k=10} & \textbf{k=20} & \textbf{k=30} \\
	    \midrule
	    sonar  & 0.533981 & 0.533981 & 0.533981\\
	    splice & 0.486437 & 0.496092 & 0.503908\\
	    \bottomrule
  		\end{tabular}
  		\caption{ISOMAP降维结果}\label{table:1}
	\end{table}

	其中ISOMAP降维的伪代码如算法$3$所示:
	\begin{algorithm}[htbp]  
        \caption{ISOMAP算法}  
        \begin{algorithmic}[1] %每行显示行号  
 			\Require 训练集矩阵$\mathbf{X}=\{\mathbf{x_1}, \mathbf{x_2},\dots,\mathbf{x_m}\}$,低维空间维数$d$,近邻参数k
	        \Ensure  矩阵$\mathbf{X}$在低维空间的投影;
	        \For{$i = 1 \to m$}
	        	\State 计算$\mathbf{x_i}$的$k$近邻;
	        	\State 将$\mathbf{x_i}$与$k$近邻点的距离设为欧式距离,与其余点的距离设为无穷大(很大的值即可);
	        \EndFor
	        \State 利用最短距离算法计算任意两个样本间的最短距离矩阵$\mathbf{D}$:$d_{i,j} = dist(\mathbf{x_i}, \mathbf{x_j})$;
	        \State 将$\mathbf{D}$作为MDS算法的输入;
            \State \Return{MDS算法的输出}
        \end{algorithmic}  
    \end{algorithm} 

	\begin{algorithm}[htbp]  
	    \caption{MDS算法}  
	    \begin{algorithmic}[1] %每行显示行号  
	 		\Require 距离矩阵$\mathbf{D}$,其元素$dist_{ij}$为样本$\mathbf{x_i}$到样本$\mathbf{x_j}$的距离;低维空间维数$d$
		    \Ensure 降维后的矩阵
		    \State 求矩阵$\mathbf{D}$的内积矩阵$\mathbf{B}$;
		    \State 对内积矩阵$\mathbf{B}$做特征值分解;
		    \State 取$\mathbf{A}$为$d$个最大的特征值所构成的对角矩阵,$\mathbf{V}$为相对应的特征向量矩阵;
	        \State \Return{$\mathbf{VA}^{1/2}$}
	   \end{algorithmic}  
	\end{algorithm} 
\end{enumerate}

\end{solution}
\newpage
\begin{thebibliography}{}
\bibitem[1]{1} https://blog.csdn.net/huangjx36/article/details/78056375
\bibitem[2]{2} 机器学习,周志华
\bibitem[3]{3} http://www.doc88.com/p-1408676906732.html
\bibitem[4]{4} Nigam, Kamal, and Rayid Ghani. "Analyzing the effectiveness and applicability of co-training." Proceedings of the ninth international conference on Information and knowledge management. ACM, 2000.
\bibitem[5]{5} 苏金树,张博锋.基于机器学习的文本分类技术研究进展[J].软件学报,2006(17):1848-1859
\bibitem[6]{6} https://www.xzbu.com/8/view-4873105.htm
\end{thebibliography}

\end{document}